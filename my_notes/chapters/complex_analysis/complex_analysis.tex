\documentclass{caesar_book}
\usepackage[english]{babel}
\usepackage{todonotes}
% figure support
\usepackage{import}
\usepackage{xifthen}
\usepackage{pdfpages}
\usepackage{amssymb}
\usepackage{amsmath, amssymb}
\usepackage{transparent}
\usepackage{textcomp}
\pdfsuppresswarningpagegroup=1

\title{My Notes}
\author{Abdalaziz Rashid}

\newcommand{\im}{\rm{i}}
%\newcommand{\R}{\mathbb{R}}
%\newcommand{\C}{\mathbb{C}}
\newcommand{\incfig}[1]{%
	\def\svgwidth{\columnwidth}
	\import{./figures/}{#1.pdf_tex}
}
\usepackage[most]{tcolorbox}
\usepackage{chngcntr}
\usepackage{lipsum}

% counters
\newcounter{theorem}
\newcounter{definition}
\counterwithin{theorem}{chapter}
\counterwithin{definition}{chapter}

% names for the structures
\newcommand\theoname{Theorem}
\newcommand\defname{Definition}

\makeatletter

% environment for theorems
\newtcolorbox{theorem}[1][]{
breakable,
enhanced,
colback=blue!05,
colframe=orange,
top=\baselineskip,
enlarge top by=\topsep,
overlay unbroken and first={
  \node[xshift=10pt,thick,draw=blue,fill=blue!20,rounded corners,anchor=west] at (frame.north west) %
  {\refstepcounter{theorem}\strut{\bfseries\theoname~\thetheorem}\if#1\@empty\relax\relax\else~(#1)\fi};
  }
}

% environment for lemas
\newtcolorbox{definition}[1][]{
breakable,
enhanced,
colback=red!05,
colframe=red!70!black,
top=\baselineskip,
enlarge top by=\topsep,
overlay unbroken and first={
  \node[thick,draw=green!40!black,fill=green!20,rounded corners] at (frame.north) %
   {\refstepcounter{definition}\strut{\bfseries\defname~\thedefinition}\if#1\@empty\relax\relax\else~(#1)\fi};
 }
}


%\usepackage[svgnames]{xcolor}
\usepackage[most]{tcolorbox}
\usetikzlibrary{shadows}

\newcounter{exa}

\tcbset{
myexample/.style={
  enhanced,
  colback=yellow!10!white,
  colframe=red!50!black,
  fonttitle=\scshape,
  titlerule=0pt,
  title={\refstepcounter{exa}example~\theexa.},
  title style={fill=yellow!10!white},
  coltitle=red!50!black,
  drop shadow,
  highlight math style={reset,colback=LightBlue!50!white,colframe=Navy}
  }
}

\newtcolorbox{texample}{myexample}

\input{math_commands.tex}
\makeatother

\begin{document}
% no page numbering in front matter
\frontmatter
% generate the title page
% \maketitlepage
% show the table of contents
% \listoftodos
% \tableofcontents
% start to number the pages
\mainmatter
% The first chapter with annotation and citations
\chapter{Complex Numbers}

\section{Topology In The Plane}
\todo{Title should be rethinked \ldots}
\subsection{Modulus of complex numbers}
Say we have circle in the complex plane with center $z_0 = x_0 +y_0 \im$ and a radius $r$ where 
    \begin{align*}
        &B_r(z_0) = {z \in \sC: \, \|z-z_0\| < r} &&
            \text{disk of radius $r$, centered at $z_0$}\\
        &K_r(z_0) = {z \in \sC: \, \|z-z_0\| = 0} &&
            \text{circle of radius $r$, centered  $z_0$}
    \end{align*}
where $d$ is the distance between two points given by the Pythagorean theorem
    \begin{align*}
     d &= \sqrt{\left( x-x_0 \right)^2+\left( y-y_0 \right)^2 }\\
       &= \|\left( x-x_0 \right) + \left( y - y_0 \right)\im \|\\
       &= \|z-z_0\|
    \end{align*}

\begin{figure}[ht]
    \centering
    \incfig{distance}
    \caption{Distance between two points}
    \label{fig:distance}
\end{figure}

\subsection{Interior Points and Boundary Points}
\begin{definition}[Interior Points]
    Let $E \subset \sC$.  $A$ point  $z_0$ is an interior point of  $E$ if there
    is some  $r > 0$ such that  $B_r(Z_0) \subset E$.
\end{definition}

\begin{definition}[Boundary Points]
    Let $E \subset \sC$. A point $b$ is a boundary point of  $E$ if every disk around  $b$ contains a point in  $E$ and a point not in  $E$.\\
    The boundary of the set  $E \subset \sC$, $\partial E$, is the set of all boundary points of $E$
\end{definition}

\begin{figure}[ht]
    \centering
    \incfig{interior_points}
    \caption{Interior and boundary points of the set $S$: black disk represent the interior points and the red disk represent the boundary points}
    \label{fig:interior_points}
\end{figure}
\begin{definition}[Open and closed sets]
    A set $U \subset \sC$ is \emph{open} if every one of its points is an
    interior point. \\
    A set $A \subset \sC$ is \emph{closed} if it contains all of its boundary
    points.
\end{definition}

\subsection{Open and Closed Sets}
\begin{itemize}
    \item $\{z \in \sC:\, \|z-z_0\| < r \}$ and $\{z \in \sC:\, \|z-z_0\|>r \}$ are open.
    \item $\{z \in \sC:\, \|z-z_0\| \le r \}$ and $\{z \in \sC:\, \|z-z_0\| = r \}$ are closed.
    \item $\sC$ and $\emptyset$ are open and closed.
    \item $\{z \in \sC:\, \|z-z_0\| \le r \} \cup  \{z \in \sC:\, \|z-z_0\| = r \text{ and } \rm{Im}(z-z_0) > 0 \}$ is neither open nor closed.
\end{itemize}

The set $\{ z \in, \|z\| < 1\} \cup \{z \in \sC, \|z\|>2\}$ is also open and
closed, in fact this set is $\emptyset$ as there are no point in the complex
plane whose modulus is both $<1$ and  $>2$ at the same time. For the sake of
contradiction say that  
    $\exists z \in \{z \in \sC, \|z\| < 1 \} \cup \{z \in \sC, \|z\| > 2\}$. This implies we have $\|z\|<1$ and $\|z>2\| \iff 2 < \|z\| < 1$! thus it is a false statement, because $\emptyset = \overline{\sC}$ is open and close.

A set can be neither open nor closed when it does not contain all of its
boundary points. as for instance, $1+\im$ is a boundary point but does not
belong to the set. However, $0$ is an element of the set but not an interior
point,since for ever $\epsilon > 0$, the disk  $B_{\epsilon}(0)$ is not
entirely contained in the set.Thus the set is neither open nor closed. 

The set $\sC$ is open and closed, since  $\overline{\sC}=\sC$, so $\sC$ contains all of
its boundary points, thus it is closed. And $\forall z \in \sC$, there is an
instance $B_1(z) \subset \sC$, this implies the set is also open.

\subsection{Closure and Interior of a Set}
A closure of a set the itself with all its boundary points, and with the same intuition we can make a set open (interior) by excluding all its boundary points.
\begin{definition}
    Let $E \in \sC$.\\
    The \emph{closure} of $E$ is the set  $E$ together with all its boundary points: $\overline{E} = E \cup \partial E$.
The \emph{interior} of $E$,  $\mathring{E}$ is the set of all interior point of $E$.
\end{definition}

\begin{texample}
    \begin{itemize}
        \item $\overline{B_r(z_0)} = B_r(z_0) \cup K_r(z_0) = \{z \in \sC:
            \|z-z_0\| \le r\}  $.\\
            The closure of of the disk $r$ centered at  $z_0$ is the disk $B_r$
            of radius  $r$ and the circle $K_r$.
        \item $\overline{K_r(z_0)=K_r(z_0)}$.\\
            The closure of the circle is the circle itself, since it contains
            all its boundary points.
        \item $\overline{B_r(z_0) \setminus  \{z_0\} } = \{z \in \sC :
            \|z-z_0\| \le r\}. $\\
            The disk $B_r$ excluding its center, when taking its closure, the
            closure will bring the center back in the set. 
        \item $E=\{z \in \sC : \|z-z_0\| \le r\} $, $\mathring{E}=B_r(z_0)$.\\
            The interior of the set  $E$ is disk  $B_r$ of radius  $r$.
        \item  $E = K_r(z_0)$,  $\mathring{E} = \emptyset$.\\            There is no point on the circle  $K_r$ that we can fit a disk with the center  $r$.
    \end{itemize} 
\end{texample}

\subsection{Connectedness}
Informally the set is connected if it is a single piece.
\begin{definition}[Connectedness of sets]
    Two sets $X, Y \in \sC$ are separated if there are disjoint open set $U, V$ so that  $X \subset U$ and  $Y \subset V$.\\
   A set  $W \in \sC$ is connected if it is impossible to find two separated non-empty sets whose union equals $W$.
        
\end{definition}
\marginpar{Two sets is said to be disjoint if $A \cap B = \emptyset$.}

\begin{texample}
$X = \left[0, \right) \text{ and } Y=\left(1, 2\right)$
are separated. Say that $U=B_1(0)$ and  $V=B_1(2)$. Thus  $X \cup Y=\left[ 0,1 \right] \setminus \{1\}$ is not connected.
\begin{center}
    \incfig{connecteness_of_sets}
\end{center}
\end{texample}
\marginpar{It is hard to check whether a set is connected.}

For the open set it is easier to check whether a set is connected or not:
\begin{definition}
   Let $G$  be an open set in $\sC$. Then $G$ is connected if and only of any two points in  $G$ can be joined  $G$ by successive line segments.
\end{definition}[Connectedness of open set]
\begin{figure}[ht]
    \centering
    \incfig{connectedness_open_set}
    \caption{Connectedness in open set: The theorem above tell us that if we can connect two points in $G$ by line segments then the set is connected}
    \label{fig:connectedness_open_set}
\end{figure}

\begin{definition}[Bounded sets]
    A set $A \in \sC$ is \emph{bounded} if there exists a number $R>0$ such that $A \subset B_R(0)$. If no such  $R$ exists then $A$ is called \emph{unbounded}.
\end{definition}
\begin{figure}[ht]
    \centering
    \incfig{bounded_set}
    \caption{Bounded set $G$: The circle has large enough radius $R$ to include all the set}
    \label{fig:bounded_set}
\end{figure}

\marginpar{In $\sR$, there are to directions for infinity $\mp \infty$.\\ And $\in \sC$ there is only one $\infty$ which can be attained in many directions}

\begin{figure}[ht]
    \centering
    \incfig{unbounded_set}
    \caption{Unbounded set: There is no circle of radius $R$ can bound all the set  $G$. Such unbounded set is the complex plane or the real plane all these are unbounded sets}
    \label{fig:unbounded_set}
\end{figure}
\begin{texample}
    Which of the fallowing radii $R$ satisfy that the disk or radius 2, centered at  $\im$, is contained in the disk of radius  $R$, centered at the origin?

    The disk or radius  2, centered at $\im$ is the set  $\{z \in \sC, \|z-i\| < 2\} $.\\
    Let $z\in \{z\in \sC, \|z-i\|< 2\} $. Using triangle inequality, we have $\|z\|=\|z-i+i\|\le \|z-i\|+ \|i\|$. But we know $\|z-i\|<2$, and $\|i\|=1$.\\
    so $\|z\| < 2 + 1 = 3$. Thus for $R\ge 3$, if $z \in \{z \in \sC, \|z-i\|<2\}$ thus $\{z\in\sC,\|z-i<2\|\}$ is contained in all disks centered at the origin, of radii $R\ge 3$.
        
\end{texample}
\end{document}

